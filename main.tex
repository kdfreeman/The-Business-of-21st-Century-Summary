\documentclass[a4paper, 12pt]{report}

%%%%%%%%%%%%
% Packages %
%%%%%%%%%%%%

\usepackage[english]{babel}
\usepackage[noheader]{packages/sleek}
\usepackage{packages/sleek-title}
\usepackage{packages/sleek-theorems}
\usepackage{packages/sleek-listings}

%%%%%%%%%%%%%%
% Title-page %
%%%%%%%%%%%%%%

%\logo{./resources/pdf/logo.pdf}
\institute{The Business of $21^{st}$ century}
\faculty{Robert T. Kiyosaki}
%\department{Department of Anything but Psychology}
\title{A Short Summary}
\subtitle{with key notes}
\author{\textit{Author}\\Kapil \textsc{Dahal}}
%\supervisor{Linus \textsc{Torvalds}}
%\context{Well, I was bored...}
\date{\today}

%%%%%%%%%%%%%%%%
% Bibliography %
%%%%%%%%%%%%%%%%

\addbibresource{./resources/bib/references.bib}

%%%%%%%%%%
% Others %
%%%%%%%%%%

\lstdefinestyle{latex}{
    language=TeX,
    style=default,
    %%%%%
    commentstyle=\ForestGreen,
    keywordstyle=\TrueBlue,
    stringstyle=\VeronicaPurple,
    emphstyle=\TrueBlue,
    %%%%%
    emph={LaTeX, usepackage, textit, textbf, textsc}
}

\FrameTBStyle{latex}

\def\tbs{\textbackslash}

%%%%%%%%%%%%
% Document %
%%%%%%%%%%%%

\begin{document}
    \maketitle
    \romantableofcontents

    \chapter{Take Control Of Your Future}

    We are living in a bad time right now. Unemployment rate is rising. Underemployment is also being a main issue. Each and every sector is at risk.
    
    People lost a lot of money in market crash of 2008. Your income is always at risk. Most of the Americans lost more than of their retirement portfolio in the crash of 2008.  



    \begin{lstlisting}[style=latexFrameTB]
                A financial disaster is coming, caused by education system's failure to adequately provide a realistic financial education program to students.
                
                The number of people living officially below poverty line is increasing and also the people working of age above 65 is increasing.
    \end{lstlisting}
    
    The percentage of Americans who owns the home is dropping. Middle-class are dropping. And, the people living in poverty is increasing.
    
    \begin{lstlisting}[style=latexFrameTB]
               Your Job is not going to take care of you. Government is not going to take care of you.
               
               Job Security, Pension, Social Security are obsolete ideas for securing future. People realised that it may not be safe from the major economic crisis.
               
               If you want a solid future, you need to create it. You can take charge of your future only when you take control of your income source. That suggests, you need your own business.
    \end{lstlisting}


    \chapter{The Silver Lining }

    \section{\texttt{Employment Mythology}}

    People are brainwashed with the concept of employment is the best way of securing the future. The concept of retirement was injected in people's mind.
    
    \begin{lstlisting}[style=latexFrameTB]
               "Employment for life" was a slogan by IBM in in 60s. Now we know where IBM is.
               
               In 50s, there was a popular slogan in America. "As GM goes, so goes the nation..". Now, GM is no longer in that status.
    \end{lstlisting}
    \section{\texttt{Entrepreneurial Fever}}
    Employment is not a bad thing. But, it is the way of generating income that is too limited. 
    
    \begin{lstlisting}[style=latexFrameTB]
               Entrepreneurial fever is rising day by day. When economy slows down, entrepreneur rises. It is the best time to evolve as an entrepreneur.
               
               US household data shows that the net worth of entrepreneur is five times more than that of conventional employees.
               
               Entrepreneurship is the key to solve the current economic crisis.
    \end{lstlisting}
    
   \chapter{Where Do You Live?}
   You have been working hard for years climbing the ladder. But, it doesn't even matter how fast or high you climb on the ladder, if it's leaning against the wrong wall.
   
    \section{How do you make the money you make?}
    The quality of money you make is more than the quantity of money you make. It is more important how the money comes from, or the source of money. There are four distinct sources of cashflow. They are:
    \begin{itemize}
        \item \textbf{E} = Employee
        \item \textbf{S} = Self-Employed
        \item \textbf{B} = Business Owner
        \item \textbf{I} = Investor
        
    \end{itemize}
    
    \subsection{The E Quadrant}
     \begin{lstlisting}[style=latexFrameTB]
               Our Education system train us to live our life in E Quadrant.
    \end{lstlisting}
    \subsection{The S Quadrant}
     \begin{lstlisting}[style=latexFrameTB]
               To get more freedom and self-determination, people move from S quadrant to E quadrant. It includes huge range of earning power.
               
               But still, S is typically a slavery. You don't own a business, but your business owns you.
    \end{lstlisting}
   $\newline$
    \subsection{The B Quadrant}
     \begin{lstlisting}[style=latexFrameTB]
               This is where people create big business. The difference between S and B is that in B, your business works for you.
    \end{lstlisting}
    \subsection{The I Quadrant}
     \begin{lstlisting}[style=latexFrameTB]
               Becoming an Investor.
    \end{lstlisting}
    Breaking away from typical Job structures and creating your own stream of income puts you in best position to tackle an economic crisis, simply because you are no longer in control of your boss or economy to determine your annual income.
    \begin{framedquest*}
       \textbf{Which quadrant do you want to live?}
    \end{framedquest*} 
    
    \chapter{Your Core Financial Values}
    The four different quadrants are four different mindsets. You have to find out your strengths, weakness and central interests.
    \begin{lstlisting}[style=latexFrameTB]
               Shifting from E or S quadrant to B quadrant is not as simple as you think like changing an address in the post office. You are going to change who you are. Your definition.
               
               You can tell which quadrant people are living in by listening to their words.
               
               
    \end{lstlisting}
    \section{The E Quadrant Values}
    \begin{lstlisting}[style=latexFrameTB]
               I am looking for a safe, secure job with good pay and excellent benefits.
    \end{lstlisting}
    \section{The S Quadrant Values}
    \begin{lstlisting}[style=latexFrameTB]
               If you want something done right, do it yourself. The core value is independence.
               
               S quadrant includes small business owners, mom and pop business, specialists, and consultants.
    \end{lstlisting}
    \begin{note}People from S or E quadrant, who is having difficulty to jump into B quadrant, is a person with great technical or management skills, but with little leadership skills. To make jump from S or E to B, you have to make a quantum jump in leadership skills.
    \end{note}
    $\newline$
    \section{The B Quadrant Values}
    \begin{lstlisting}[style=latexFrameTB]
               I am looking for best people in my team. The core value of people in B quadrant is wealth building.
               
               B quadrant people person wants to build a team of other people who are best in their fields.
               
               In B quadrant, you are dealing with much smarter, more experienced, and more capable people than you are.
    \end{lstlisting}
    \begin{framedquest*}
       \textbf{If I stop working today, what will be my income inflow in each of the E, S and B quadrant?}
    \end{framedquest*}
    The answer is simple. Income stops in E and S if you stop working. But in B quadrant, money will continue to come in although you stop working.
    \section{The I Quadrant Values}
    \begin{lstlisting}[style=latexFrameTB]
               What's my return on investment?
               
               People in I quadrant wants financial freedom.
    \end{lstlisting}
    
    \begin{note}\textbf{If you want to get rich, you are going to have to move. You don't need a new job. You need a new address.}
    \end{note}
    
    \chapter{The Mindset Of An Entrepreneur}
    As an entrepreneur, you can raise capital from three group of people; customers, investor, and employees.
    \begin{lstlisting}[style=latexFrameTB]
               You don't have to raise capital yourself to create your business, because that has already been done for you. All you have to do is build your business.
               
               If you can't get your employees to produce at least ten times more than you pay them, you're out of business and no need to raise money.
               
               Entrepreneurs make things happen.
    \end{lstlisting}
    \begin{framedquest*}
       \textbf{What does it take to be an Entrepreneur?}
    \end{framedquest*}
    \begin{lstlisting}[style=latexFrameTB]
               It takes courage to discover, develop and donate your genius to the world.
               
               Your mind have infinite potential. It's your doubts that are limiting.
    \end{lstlisting}
    \begin{framedquest*}
       \textbf{What do you want to be when you grow up?}
    \end{framedquest*}
    \begin{note}\textbf{One of the beauties of the 21st century is that all the groundwork of the business is done for you.}
    \end{note}
    
    \chapter{It's Time to Take Control!}
    
    People often say, "It takes money to make money". But that is totally incorrect. That is the mindset of small-minded people. Also, one doesn't require a good formal education to build wealth.
     \begin{framedquest*}
       \textbf{What do you need to become financially free?}
    \end{framedquest*}
    \begin{lstlisting}[style=latexFrameTB]
               All you need is a dream, a lot of determination, a willingness to learn quickly, and an understanding of which portion of cashflow you currently are in.
    \end{lstlisting}
    \section{Hard Work Will Not Make You Rich}
    We all have seen someone in our circle, either friends or relatives, who has worked very hard his/her entire life and living a life with economic crisis.
    \begin{lstlisting}[style=latexFrameTB]
               Working hard at making money will never create wealth.
    \end{lstlisting}
    \subsection{The Problem}
    \begin{lstlisting}[style=latexFrameTB]
               Building the business is the way most of the rich become very rich. But starting a business is a risk to many people. The failure rate for new business is about 90%.
               
    \end{lstlisting}
    \subsection{What about the franchise?}
    \begin{lstlisting}[style=latexFrameTB]
               A franchise is not risk at all. But still, it takes a lot of money to buy and maintain it. There are monthly payments to headquarters of franchise for training, advertising and support.
               
               In theory franchise is a great idea. But, in reality it's a gamble.
    \end{lstlisting}
    
    \section{The Power Of Passive Income}
    Income that continues coming in, over and over, long after you finished expending the effort and capital it took to create the source of income is the Passive Income.
    
    \begin{lstlisting}[style=latexFrameTB]
               One business model creates passive income, but requires relatively little cash investment to start up. That business model is called Network Marketing.
    \end{lstlisting}
    \chapter{My Years In Business}
    \section{The Opening Of a Mind}
     \begin{lstlisting}[style=latexFrameTB]
               Personal success is fulfilling. But, it's much more satisfying when you can create many others create their own success as well.
    \end{lstlisting}
    
    \begin{framedquest*}
       \textbf{What exactly is Network Marketing?}
    \end{framedquest*}
    \begin{lstlisting}[style=latexFrameTB]
               The idea is simple. Instead of spending tons of money on all sorts of professional agencies and marketing channels to promote products and services, pay the people who love them most to just tell others about them. It is like harnessing the power of word-of-mouth --person-to-person relationship.
    \end{lstlisting}
    
    \begin{framedquest*}
       \textbf{What others say about Network Marketing?}
    \end{framedquest*}
    \begin{lstlisting}[style=latexFrameTB]
               Today, network marketing is recognized by many experts and accomplished business people as one of the fastest-growing business models in the world.
    \end{lstlisting}
    \chapter{It's Not About Income- It's About Assets That Generate Income}
    Many people don't understand the value of network marketing. Many of those who are actually involved with it themselves don't fully know their potential as well.
    \begin{lstlisting}[style=latexFrameTB]
               The B and I quadrants are not about earning more income; they are about owing assets that generate income.
    \end{lstlisting}
    \section{The Truth About Your House}
    \begin{lstlisting}[style=latexFrameTB]
               House is not an asset. It's a liability. It is not generating money that goes to your pocket. But, instead it is taking money from your pocket for various causes.
    \end{lstlisting}
     \begin{framedquest*}
       \textbf{How much money does your house bring, month in and month out?}
       
    \end{framedquest*}
    \section{How to Know Your Asset from a Hole in the Ground?}
     \begin{lstlisting}[style=latexFrameTB]
               An asset is something that works for you. So, you don't have to work for rest of your life.
               
               Owning a business is owning an asset. When you build a network marketing business, you're not only learning critical life skills, you are also building a genuine asset for yourself.
    \end{lstlisting}
     \section{Network Marketing Is Not Selling Products or Earning Income!}
     \begin{lstlisting}[style=latexFrameTB]
               In network marketing, the whole point is not to sell a product, but build a network, an army of people who are all representing that same product or service to share with others.
               
               It's about building an assets.
               
               Actually, it's about building eight assets, all at the same time.
    \end{lstlisting}
    
   \chapter{Asset \#1: A Real World-Business Education}
   \section{Three Kinds of Education}
   If you want to be financially successful, there are three different kinds of education you should require: scholastic, professional and financial education.
   \subsection{Scholastic Education}
     \begin{lstlisting}[style=latexFrameTB]
               It teaches you how to read, write and do math.
    \end{lstlisting}
    \subsection{Professional Education}
     \begin{lstlisting}[style=latexFrameTB]
               It teaches you how to work for money. It prepares you for life in E and S quadrants.
    \end{lstlisting}
    \subsection{Financial Education}
     \begin{lstlisting}[style=latexFrameTB]
               It teaches you to have money work for you rather than to have you work for money.
    \end{lstlisting}
    
    
    \section{The Important Skills}
    Being an entrepreneur is not an easy task. That's why lots of people live their life in E and S quadrant.
    \subsection{Tax Advantages}
    Most of the people have at least a vague idea about how rich enjoys tax advantages, and they do not. Since they are living in E quadrant, they don't know how it works.
   
    \begin{lstlisting}[style=latexFrameTB]
               By starting a network marketing business in your spare time and keeping your regular job, you begin to gain the tax advantages of the rich. 
               
               One of the beauties of Network Marketing business is that it starts to show your life in the B quadrant.
    \end{lstlisting}
    
    \section{Life Skills}
    \begin{lstlisting}[style=latexFrameTB]
               The key to long-term success in life is your education and skills, your life experiences, and most of all, your personal character.
               
               Overcome self-doubt, shyness, and fear of rejection.
    \end{lstlisting}
    Below are some of the critical skills, the real world of network marketing teaches:
    \begin{itemize}
        \item \textbf{An attitude of success}
         \item \textbf{Dressing for success}
          \item \textbf{Overcoming personal fears, doubts, and lack of confidence}
           \item \textbf{Overcoming the fear of rejection}
            \item \textbf{Communication skills}
             \item \textbf{People skills}
              \item \textbf{Time-management skills}
               \item \textbf{Accountability skills}
                \item \textbf{Practical goal setting}
                 \item \textbf{Money management skills}
                  \item \textbf{Investing skills} 
        
        
    \end{itemize}
     \begin{lstlisting}[style=latexFrameTB]
              Network marketing is a real-world business school for people who want to learn the real-world skills of an entrepreneur, rather than the skills of an employee.
              
              In network marketing, the training is more than theory. It's experimental.
    \end{lstlisting}
    
    \chapter{Asset \#2: A Profitable Path of Personal Development}
    It's about changing who you are.
    \begin{lstlisting}[style=latexFrameTB]
              Don't quit. It's a sign of failure.
              
              The winner is also inside you. The loser is also inside you. It's your choice. You become what you choose.
              
              Network marketing gives the opportunity to face your fears, deals with them, overcome them, and bring out the winner that you have living inside you.
    \end{lstlisting}
    
    \chapter{Asset \#3 : A Circle of Friends Who Share Your Dreams And Values}
    If you want to create a different economy in your life, you may need to get new friends more than you need to get a new job.
    \begin{lstlisting}[style=latexFrameTB]
              If you are considering building your own business, you need to be acutely aware of who you're spending your time with and who your teachers are.
              
              Network marketing not only provides a great business education, it also provides a whole-new world of friends-friends who are going in same direction as you are and share the same core values as you do.
    \end{lstlisting}
    
    \chapter{Asset \#4 : The Power Of Your Own Network}
    \begin{lstlisting}[style=latexFrameTB]
              The power is not in the product;the power is in the network. If you want to become rich, the best strategy is to find a way to build a strong, viable and growing network.
              
              The richest people in the world build networks. Everyone else looks for work.
    \end{lstlisting}
    \section{Metcalfe's Law}
    Robert Metcalfe, the founder of 3Com and one of the creator of Ethernet, is credited with creating an equation that defines the values of Networks:
    \[V=N^2\]
    
    \begin{note}
        A network's economic value equals the number of network's users squared. As you add users, it's value increases automatically. 
    \end{note}
    \begin{lstlisting}[style=latexFrameTB]
             The economic value of network goes up exponentially, not numerically.
              
             To harness the power of metcalfe's law, you have to grow the network by duplicating yourself in someone else just like you: a partner. 
              
             The moment there are two of you, the economic value of your network is squared.
    \end{lstlisting}
    
    \chapter{Asset \#5: A Duplicable, Fully Scalable Business}
    It is not a business for those who are gifted in sales. Forget everything you know about selling.
    
    \begin{lstlisting}[style=latexFrameTB]
             The key to success in network marketing is what you can duplicate.
             
             It's not a question of what you can do. It's what you can do and then what others can do, too.
    \end{lstlisting}
    \section{Information Tools For Infinite Scalability}
    The power of your business is in it's scalability. A business that is scalable means a business that can operate in any scale.
    
    \begin{lstlisting}[style=latexFrameTB]
             Low-cost, high-quality CDs, DVDs, and online presentations have made possible the dream of a fully democratic and fully scalable network marketing operation, creating a business model that has allowed millions to gain access and excel.
    \end{lstlisting}
    
    \chapter{Asset \#6: Incomparable Leadership Skills}
     \begin{lstlisting}[style=latexFrameTB]
             Leadership is the force that makes it all come together. Leadership is what builds great businesses.
             
             The power to make things happen through the sheer force of the vision you share. Genuine leaders can move mountains.
             
             Money doesn't goes to the business with the best products or service. Money flows to the business with the best leaders.
    \end{lstlisting}
    All the great leaders have been master storytellers who were able to communicate the vision in such a vivid way that others saw it too.
    \begin{lstlisting}[style=latexFrameTB]
             Network marketing tends to develop the type of leader who influences other by being a great teacher, teaching other's to fulfil their life's dreams by teaching others to go for their dreams.
    \end{lstlisting}
    \section{The Four Elements of Leadership}
    Below are the four elements of leadership skills:
    \begin{itemize}
        \item \textbf{Mental, Emotional, Spiritual, Physical} 
       
        
    \end{itemize}
    \begin{lstlisting}[style=latexFrameTB]
            Mind; Spirit; Body; Emotions
            
            If you cannot control these four aspects of yourself, then you will fail.
            
            If you are not able to develop these aspects in your employees, in doing so helping them to make effective leaders, you will fail.
    \end{lstlisting}
    
    
    \chapter{Asset \#7: A Mechanism for Genuine Wealth Creation}
     \begin{lstlisting}[style=latexFrameTB]
            Wealth is not the same thing as money. Wealth is not measured by the size of income. Wealth is measured in time.
            
            Wealth is measured by the richness of your life experience today plus the number of days into the future that you have the capacity to continue living at that level of experience.
    \end{lstlisting}
    \section{Four Step Path to Financial Freedom}
    Below are the steps to reach the financial freedom:
    \begin{itemize}
        \item \textbf{Build a Business}
         \item \textbf{Reinvest in a Business}
          \item \textbf{Invest in real estate} 
           \item \textbf{Let your assets buy luxuries} 
        
    \end{itemize}
    \subsection{Build a Business}
    \begin{lstlisting}[style=latexFrameTB]
            Building a business allows you to generate a lot of money. The tax law is favorable to People in B quadrant.
    \end{lstlisting}
    \subsection{Reinvest in a Business}
    \begin{lstlisting}[style=latexFrameTB]
            The reason so many people fail to achive great wealth in any business is simply that they fail to reinvest continually in business.
    \end{lstlisting}
    \subsection{Invest in a real estate}
    \begin{lstlisting}[style=latexFrameTB]
            Use your money to build an income generating asset. One of the income-generating asset is real estate.
            
            Tax laws are written in favor of business owners who invest in real estate.
            
            Banker loves to lend money to real estate.
    \end{lstlisting}
    \subsection{Let your assets buy you luxuries}
    \begin{lstlisting}[style=latexFrameTB]
            Luxury mean something that you want and enjoy. That exists beyond what you need.
            
            Use your income to build your assets. Later, let them buy you luxuries.
            
    \end{lstlisting}
    \chapter{Asset \# 8: Big Dreams and the Capacity to Live Them}
    One of the most valuable things about network marketing companies is that they stress the importance of going for your dreams. They don't just want to to have dreams; they want you to live those dreams.
    
     \begin{lstlisting}[style=latexFrameTB]
            When you make a habit of asking yourself, "How can I afford that?", you train yourself to dream bigger and bigger dreams, not only to have those dreams but to believe that you can make them come true.
            
            It is striving, learning, and doing your best to develop your personal power to be able to afford the big house and who you become in the process that are important.
    \end{lstlisting}
    There are basically five kinds of dreamers. They are:
    \begin{itemize}
        \item \textbf{Those who dream in the past.}
        \item \textbf{Those who dream only small dreams.}
        \item \textbf{Those who achieve a dream, and then live bored.}
        \item \textbf{Those who dream big dreams, but with no plan on how to go about achieving them, so end up with nothing.}
        \item \textbf{Those who dream big, achieve those dreams, and go on to dream even bigger dreams!.}
        
    \end{itemize}
     \begin{lstlisting}[style=latexFrameTB]
           Big people have big dreams and small people have small dreams. If you want to change who you are, begin by changing the size of your dream.
            
    \end{lstlisting}
    \begin{framedquest*}
       \textbf{Which category of dreamer do you want to become ?}
       
    \end{framedquest*}
    
    
    
    

   


\end{document}
